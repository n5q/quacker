\documentclass[11pt]{article}

\usepackage{courier} \usepackage{enumerate} \usepackage{enumitem}
\usepackage{float} \usepackage[margin=1in]{geometry} \usepackage{graphicx}
\usepackage{listings} \usepackage{xcolor} \usepackage{amsmath}
\usepackage{amssymb}

\lstset{ language=Python, basicstyle=\ttfamily\small, keywordstyle=\color{blue},
commentstyle=\color{gray}, stringstyle=\color{red}, showstringspaces=false,
identifierstyle=\color{black}, breaklines=true, columns=fullflexible,
backgroundcolor=\color{white}, % you can change the background color }

\begin{document}

\title{\huge Quacker} \author{Omar Mahmoud, Nasif Qadri, Yousef Moussa\\[0.5cm]
Design Document - CMPUT 291 Mini Project 1\\} \date{2024-11-14}

\maketitle % \newpage % \begin{abstract} % abstarct % \end{abstract}

% The design document should include (a) a general overview of your system with
a small user guide, (b) a detailed design of your software with a focus on the
components required to deliver the major functions of your application, (c) your
testing strategy, and (d) your group work break-down strategy. The general
overview of the system gives a high-level introduction and may include a diagram
showing the flow of data between different components; this can be useful for
both users and developers of your application. The detailed design of your
software should describe the responsibility and interface of each primary
function or class (not secondary utility functions/classes) and the structure
and relationships among them. Depending on the programming language being used,
you may have methods, functions, or classes. The testing strategy discusses your
general strategy for testing, with the scenarios being tested, the coverage of
your test cases, and (if applicable) some statistics on the number of bugs found
and the nature of those bugs. The group work strategy must list the breakdown of
the work items among partners, both the time spent (an estimate) and the
progress made by each partner, and your method of coordination to keep the
project on track. The design document should also include documentation of any
decision you have made that is not in the project specification or any coding
you have done beyond or different from what is required.

% a) \section{Overview and User Guide}

\subsection{System Overview} Quacker is a lightweight social media platform
that enables users to share short posts (\textit{Quacks}), follow other users,
and interact with users.

\subsection{Features} The core features of Quacker include: \begin{itemize}
\item \textbf{User Authentication:} Users can log in or sign up for an account.
\item \textbf{Posting Quacks:} Users can create new Quacks or reply to existing
ones. \item \textbf{Feed Management:} Users can view a personalized feed
containing Quacks from people they follow. \item \textbf{Search Functionality:}
Users can search for other users or specific Quacks by keywords or hashtags.
\item \textbf{Follow System:} Users can follow others to curate their feed.
\end{itemize}

\subsection{User Guide} This section provides a step-by-step guide to using
Quacker.

\subsection*{Getting Started} \begin{enumerate} \item Launch the Quacker
application. \item On the welcome page, choose from the following options:
\begin{itemize} \item Press \textbf{1} to log in to your account. \item Press
\textbf{2} to sign up for a new account. \item Press \textbf{3} to exit the
application. \end{itemize} \end{enumerate}

\subsection*{Creating an Account} \begin{enumerate} \item Select \textbf{2}
(Sign Up) on the start page. \item Enter your details: \begin{itemize} \item
\textbf{Name:} Your full name. \item \textbf{Email:} A valid email address.
\item \textbf{Phone Number:} A 10-digit phone number. \item \textbf{Password:} A
secure password. \end{itemize} \item If successful, you will logged in.
\end{enumerate}

\subsection*{Logging In} \begin{enumerate} \item Select \textbf{1} (Log In) on
the start page. \item Enter your \textbf{User ID} and \textbf{Password}.
\end{enumerate}

\subsection*{Using the Feed} \begin{enumerate} \item View the latest Quacks
from people you follow. \item Navigate using these options: \begin{itemize}
\item Press \textbf{1} to see more Quacks. \item Press \textbf{2} to see fewer
Quacks. \item Press \textbf{3} to search for users. \item Press \textbf{4} to
search for Quacks. \item Press \textbf{5} to reply or repost a Quack from your
feed. \item Press \textbf{6} to list your followers \item Press \textbf{7} to
create a new Quack. \item Press \textbf{8} to log out. \end{itemize}
\end{enumerate}

\subsection*{Creating a Quack} \begin{enumerate} \item Navigate to the posting
page by selecting \textbf{7} from the feed menu. \item Enter the text of your
Quack. \item Press \textbf{Enter} to post the Quack. \end{enumerate}

\subsection*{Searching for Users or Quacks} \begin{enumerate} \item Select
\textbf{3} (Search for Users) or \textbf{4} (Search for Quacks) from the feed
menu. \item Enter a search term (e.g., a name or hashtag). \item View the
results and interact with users or Quacks as desired (follow, reply, etc...).
\end{enumerate}

\subsection*{Logging Out} \begin{enumerate} \item Select \textbf{8} from the
feed menu to log out of your account. \item You will be redirected to the start
page. \end{enumerate}

\subsection*{Running the Application} \begin{enumerate} \item \textbf{Compile
the Application:} \\ Use the provided \textit{Makefile} to compile the code by
running: \begin{verbatim} make \end{verbatim} \item \textbf{Start the
Application:} \\ Execute the application with a database filename:
\begin{verbatim} build/quacker <database_filename> \end{verbatim} %
\textit{Example:} % \begin{verbatim} % build/quacker test/test.db %
\end{verbatim} % \item \textbf{Interacting with the Application:} \\ % Upon
launch, the application displays the main start page, providing options for
interacting with Quacker . \end{enumerate} % \newpage

% b) \section{Software Design} The application consists of the following key
components:

\subsection*{Quacker Class} The \textit{Quacker} class is the main entry point
of the application and manages user interactions. It provides functionality such
as: \begin{itemize} \item \textbf{User Authentication:} Handles login and signup
using methods \textit{loginPage()} and \textit{signupPage()}. \item \textbf{Feed
Management:} Displays the user feed using the \textit{mainPage()} function, and
interacts with the \textit{Pond} class for data. \item \textbf{Posting Quacks:}
Allows users to post and reply to Quacks using the \textit{postingPage()} and
\textit{replyPage()} methods. \item \textbf{Search Functionality:} Facilitates
searching for users and Quacks using \textit{searchUsersPage()} and
\textit{searchQuacksPage()}. \end{itemize}

\subsection*{Pond Class} The \textit{Pond} class handles all database
interactions and encapsulates the data logic. It includes: \begin{itemize} \item
\textbf{User Management:} Provides methods for adding users (\textit{addUser()})
and validating login credentials (\textit{checkLogin()}). \item \textbf{Quack
Management:} Handles operations like adding new Quacks (\textit{addQuack()}),
replies (\textit{addReply()}), and managing hashtags. \item \textbf{Feed
Retrieval:} Generates personalized user feeds (\textit{getFeed()}), combining
tweets and requacks in chronological order. \item \textbf{Search Operations:}
Performs database queries for searching users (\textit{searchForUsers()}) and
Quacks (\textit{searchForQuacks()}). \end{itemize}

\subsection*{Component Interactions} The application follows a structured flow
of data between the \textit{Quacker} and \textit{Pond} classes:
\begin{enumerate} \item The user interfaces with the \textit{Quacker} class,
which calls the relevant \textit{Pond} methods for database interactions. \item
The \textit{Pond} class processes SQL queries to retrieve, insert, or update
data in the database. \item Results from the \textit{Pond} class are used to
render user-friendly views in the \textit{Quacker} class. \end{enumerate}

% \newpage

% c) \section{Testing Strategy} In testing this program the Python script
\textit{populate\_db.py} was used repopulate the testing database with new
unqiue data improving the manually typed tests ran on the project and increasing
the number of tested edge cases. To generate simply run: \begin{verbatim}
python3 test/populate_db.py \end{verbatim} The script should then return the
data of a randomly generated user. This is to make it easier to locate a valid
login to use when entering the program. Example: \begin{verbatim} Random User
Data: User ID: 53 Name: Deborah Scott Email: jasminchang@example.com Phone:
4567624481 Password: E1AjR1ib*t SUCCESS! Database populated with random test
data. \end{verbatim} The test script also allows you to easily change parameters
regarding the generated data. These can be located in the defintion of
\textit{populate\_db()} and are self-explanitory as seen below:
\begin{lstlisting} def populate_db(db_name, user_count=100, tweet_count=500,
list_count=200, follow_count=300): \end{lstlisting} When it came to testing the
application, this was done manually due to the page centric format of our user
interface. The felxibilty of testing manually allowed our team to catch unique
edge cases that may have passed over an automated system. % \newpage

% d) \section{Group Work Breakdown} We give all credit to Allah SWT, all
success comes from Him alone. \end{document}