\documentclass[12pt]{article}

\usepackage{enumerate} 
\usepackage{enumitem}
\usepackage{graphicx}
\usepackage{float}
\usepackage[margin=1in]{geometry}


\begin{document}

\title{\huge Quacker}
\author{Omar Mahmoud,  Nasif Qadri,  Yousef Moussa\\[0.5cm] 
Design Document - CMPUT 291 Mini Project 1\\}
\date{2024-11-14}

\maketitle
\newpage
% \begin{abstract}
% 	abstarct
% \end{abstract}

% The design document should include (a) a general overview of your system with a small user guide, (b) a detailed design of your software with a focus on the components required to deliver the major functions of your application, (c) your testing strategy, and (d) your group work break-down strategy. The general overview of the system gives a high-level introduction and may include a diagram showing the flow of data between different components; this can be useful for both users and developers of your application. The detailed design of your software should describe the responsibility and interface of each primary function or class (not secondary utility functions/classes) and the structure and relationships among them. Depending on the programming language being used, you may have methods, functions, or classes. The testing strategy discusses your general strategy for testing, with the scenarios being tested, the coverage of your test cases, and (if applicable) some statistics on the number of bugs found and the nature of those bugs. The group work strategy must list the breakdown of the work items among partners, both the time spent (an estimate) and the progress made by each partner, and your method of coordination to keep the project on track. The design document should also include documentation of any decision you have made that is not in the project specification or any coding you have done beyond or different from what is required.

% a)
\section{Overview and User Guide}
	\subsection*{Running the Application}

	\begin{enumerate}
		\item \textbf{Compile the Application:} \\
		Use the provided \texttt{Makefile} to compile the code by running:
		\begin{verbatim}
		make
		\end{verbatim}

		\item \textbf{Start the Application:} \\
		Execute the application with a database filename:
		\begin{verbatim}
		build/quacker <database_filename>
		\end{verbatim}
		\textit{Example:}
		\begin{verbatim}
		build/quacker test/prj-sample.db
		\end{verbatim}

		\item \textbf{Interacting with the Application:} \\
		Upon launch, the application displays the main start page, providing options for interacting with Quacker .
	\end{enumerate}
\newpage

% b)
\section{Software Design}
\begin{center}
\textit{
Baila, baila conmigo\\
Baila, baila mi amor\\
Baila e mexe o umbigo\\
Que así es que se baila mejor\\
}
\end{center}
\newpage

% c)
\section{Testing Strategy}
We dont need to test our software because we are better
\newpage


% d)
\section{Group Work Breakdown}
We give all credit to Allah SWT, all success comes from Him alone.
\end{document}
